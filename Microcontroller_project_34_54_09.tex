\documentclass[12pt]{article}
\usepackage{amsmath}
\usepackage{latexsym}
\usepackage{amsfonts}
\usepackage[normalem]{ulem}
\usepackage{array}
\usepackage{amssymb}
\usepackage{graphicx}
\usepackage[backend=biber,
style=numeric,
sorting=none,
isbn=false,
doi=false,
url=false,
]{biblatex}\addbibresource{bibliography.bib}

\usepackage{subfig}
\usepackage{wrapfig}
\usepackage{wasysym}
\usepackage{enumitem}
\usepackage{adjustbox}
\usepackage{ragged2e}
\usepackage[svgnames,table]{xcolor}
\usepackage{tikz}
\usepackage{longtable}
\usepackage{changepage}
\usepackage{setspace}
\usepackage{hhline}
\usepackage{multicol}
\usepackage{tabto}
\usepackage{float}
\usepackage{multirow}
\usepackage{makecell}
\usepackage{fancyhdr}
\usepackage[toc,page]{appendix}
\usepackage[hidelinks]{hyperref}
\usetikzlibrary{shapes.symbols,shapes.geometric,shadows,arrows.meta}
\tikzset{>={Latex[width=1.5mm,length=2mm]}}
\usepackage{flowchart}\usepackage[paperheight=11.0in,paperwidth=8.5in,left=0.79in,right=0.79in,top=0.79in,bottom=0.79in,headheight=1in]{geometry}
\usepackage[utf8]{inputenc}
\usepackage[T1]{fontenc}
\TabPositions{0.49in,0.98in,1.47in,1.96in,2.45in,2.94in,3.43in,3.92in,4.41in,4.9in,5.39in,5.88in,6.37in,6.86in,}

\urlstyle{same}


\setcounter{tocdepth}{5}
\setcounter{secnumdepth}{5}


\setlistdepth{9}
\renewlist{enumerate}{enumerate}{9}
		\setlist[enumerate,1]{label=\arabic*)}
		\setlist[enumerate,2]{label=\alph*)}
		\setlist[enumerate,3]{label=(\roman*)}
		\setlist[enumerate,4]{label=(\arabic*)}
		\setlist[enumerate,5]{label=(\Alph*)}
		\setlist[enumerate,6]{label=(\Roman*)}
		\setlist[enumerate,7]{label=\arabic*}
		\setlist[enumerate,8]{label=\alph*}
		\setlist[enumerate,9]{label=\roman*}

\renewlist{itemize}{itemize}{9}
		\setlist[itemize]{label=$\cdot$}
		\setlist[itemize,1]{label=\textbullet}
		\setlist[itemize,2]{label=$\circ$}
		\setlist[itemize,3]{label=$\ast$}
		\setlist[itemize,4]{label=$\dagger$}
		\setlist[itemize,5]{label=$\triangleright$}
		\setlist[itemize,6]{label=$\bigstar$}
		\setlist[itemize,7]{label=$\blacklozenge$}
		\setlist[itemize,8]{label=$\prime$}

\setlength{\topsep}{0pt}\setlength{\parindent}{0pt}
\renewcommand{\arraystretch}{1.3}

\begin{document}

\vspace{\baselineskip}
\vspace{\baselineskip}
\vspace{\baselineskip}
\vspace{\baselineskip}
\vspace{\baselineskip}
\vspace{\baselineskip}
\hspace{4 mm}
{\fontsize{22pt}{20.2pt}\selectfont \textbf{Department Of Computer Science $\&$  Engineering}\par}\par
\vspace{\baselineskip}
\vspace{\baselineskip}
\hspace{55 mm}
{\fontsize{22pt}{26.4pt}\selectfont \textbf{University Of Dhaka}\par}\par
\vspace{\baselineskip}

\vspace{\baselineskip}
\vspace{\baselineskip}
\vspace{\baselineskip}
\vspace{\baselineskip}
\vspace{\baselineskip}
\vspace{\baselineskip}
\vspace{\baselineskip}
\hspace{51 mm}
{\fontsize{18pt}{21.6pt}\selectfont \textbf{Project Proposal Document}\par}\par
\vspace{\baselineskip}

\vspace{\baselineskip}
\hspace{58 mm}
{\fontsize{15pt}{18.0pt}\selectfont CSE-3116\  Microcontroller Lab\par}\par
\vspace{\baselineskip}

\vspace{\baselineskip}
\vspace{\baselineskip}
\vspace{\baselineskip}
\vspace{\baselineskip}

\vspace{\baselineskip}
\vspace{\baselineskip}
\vspace{\baselineskip}
\hspace{6 mm}
{\fontsize{24pt}{28.8pt}\selectfont \textbf{Project Title : Garbage Collector Robot }\par}\par

\vspace{\baselineskip}
\vspace{\baselineskip}

\vspace{\baselineskip}
\hspace{60 mm}
{\fontsize{18pt}{21.6pt}\selectfont \textcolor[HTML]{B45F06}{Project Partner Details :}\par}\par


\vspace{\baselineskip}

\hspace{60 mm}
{\fontsize{14pt}{16.8pt}\selectfont Md.\ Robiul islam Rony  Roll : 34\par}\par
	
\hspace{60 mm}
{\fontsize{14pt}{16.8pt}\selectfont Sree\ Sowmik Kumar Sarker  Roll : 54\par}\par
	
\hspace{60 mm}
{\fontsize{14pt}{16.8pt}\selectfont Afia Anjum Tamanna Roll : 09\par}\par


\vspace{\baselineskip}

\vspace{\baselineskip}

\vspace{\baselineskip}

\vspace{\baselineskip}

\vspace{\baselineskip}

\vspace{\baselineskip}

\vspace{\baselineskip}

\vspace{\baselineskip}

\vspace{\baselineskip}

\vspace{\baselineskip}

\vspace{\baselineskip}

\vspace{\baselineskip}

\vspace{\baselineskip}

\vspace{\baselineskip}

\vspace{\baselineskip}

\vspace{\baselineskip}

\vspace{\baselineskip}

\vspace{\baselineskip}

\vspace{\baselineskip}

\vspace{\baselineskip}

\vspace{\baselineskip}

\vspace{\baselineskip}

\vspace{\baselineskip}
{\fontsize{18pt}{21.6pt}\selectfont \textbf{\uline{Abstract :}}\par}\par

{\fontsize{14pt}{16.8pt}\selectfont In this project we made an autonomous system which can detect color and take garbage in a specific range. In this paper we will discuss about the theory needed behind the project. Besides, how we have done our project, result analysis, difficulties we have faced, accuracy etc . \par}\par

{\fontsize{18pt}{21.6pt}\selectfont \textbf{\uline{Description :}}\par}\par

{\fontsize{14pt}{16.8pt}\selectfont Home cleaning is much hard working task. Our main purpose of this project is to make a garbage collector robot that can make the cleaning task easy. It can move randomly in a specified area and can detect object in that area marked by a border. We used sensors to detect objects and it returns the object information. After getting the informations it picks up the object and keeps it outside the border. We used motors to moving armature to pick up the objects.\par}\par

{\fontsize{17pt}{20.4pt}\selectfont \textbf{\uline{Algorithm :} }{\fontsize{14pt}{16.8pt}\selectfont First we initialize all inputs as the robot starts from a static state.  Then in different state it do different action.

State-01 : 
	The robot is always searching something and control his area. If found black area then go back and then go left and then round and search object. Also sonar sensor always test sonar signal and if got or detect something less than 10 cm then go strictly to the object slowly. and stop in front of the object and go to step-02.

State-02:
	 Detect object''s color. If color is true or wanted color then  It go step -03. Otherwise it go back and then turn  left and then go state-01.

State-03:
	It start's sarvo1 motor down and then servo2 motor keep that means servo2 motor collect the object and then servo1 motor up and then the robot go forward untill black color detect . If black color detect the stop and go to state-04. Otherwise go to forwar.

State-04:
	Servo1 motor down and servo2 motor out function call so that it can remove the object in the outside area. Then it go back and then start state-01.

Thus, Garbage collector robot works.

\par}\par}\par

{\fontsize{19pt}{22.8pt}\selectfont \textbf{Methods :\tab}\par}\par

{\fontsize{18pt}{21.6pt}\selectfont \textbf{1. Apparatus :\tab }\par}\par

\begin{itemize}
	\item {\fontsize{14pt}{16.8pt}\selectfont DC motors for wheels.\par}\par

	\item {\fontsize{14pt}{16.8pt}\selectfont Ultrasonic sensors.\par}\par

	\item {\fontsize{14pt}{16.8pt}\selectfont IR sensors.\par}\par

	\item {\fontsize{14pt}{16.8pt}\selectfont Servo motor.\par}\par

	\item {\fontsize{14pt}{16.8pt}\selectfont Arduino board.\par}\par

	\item {\fontsize{14pt}{16.8pt}\selectfont Bread board.\par}\par

	\item {\fontsize{14pt}{16.8pt}\selectfont Wires\par}\par

	\item {\fontsize{14pt}{16.8pt}\selectfont Power source, battery etc.\par}
\end{itemize}\par

{\fontsize{18pt}{21.6pt}\selectfont \textbf{2. Software: Arduino(IDE)\tab}\par}\par

{\fontsize{17pt}{20.4pt}\selectfont \textbf{\uline{Result :} }{\fontsize{14pt}{16.8pt}\selectfont We experimented our project against different color as input. It can detect 70-80$\%$  accurately to detect. Vice versa is also true, that mean that if input is not a particular color , sometimes the next action is being started.\par}\par}\par


\vspace{\baselineskip}
{\fontsize{18pt}{21.6pt}\selectfont \textbf{\uline{Conclusion :} }{\fontsize{14pt}{16.8pt}\selectfont  As this is autonomous system almost,it can be used reality. We have learn many more things while working on this for last 2-3 months. As it can be updated adding additional feature, we 3 are interested to work on this in future.\par}\par}\par


\vspace{\baselineskip}
{\fontsize{18pt}{21.6pt}\selectfont \textbf{\uline{Acknowledgement :} }{\fontsize{14pt}{16.8pt}\selectfont  We thank Dr. shugata ahmed and Dr. md. mosaddek khan sir to instruct us throughout the project. Also, we want to thank all the author of the references we have used throughout our project.\par}\par}\par


\vspace{\baselineskip}

\vspace{\baselineskip}
\vspace{\baselineskip}
\printbibliography
\end{document}
